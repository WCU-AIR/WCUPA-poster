\documentclass[final]{beamer}

% <<< packages

% `size` and `orientation` are options for `beamerposter`.
% `scale` can be used to make the fonts bigger, as required.
\usetheme[size=a1,orientation=landscape, scale=1.0]{uvtposter}

% load math packages
\usepackage{amsmath}

% load more packages
\usepackage{hyperref}
\usepackage{booktabs}

% for displaying code (using `minted` will probably give better results)
\usepackage{listings}
\usepackage{graphicx}



% for qr code in header
% \usepackage{qrcode}

% >>>

% <<< formatting

% use numbers for footnotes inside columns
% \renewcommand\thempfootnote{\arabic{mpfootnote}}

% add custom logos
% \headerlogoleft{\includegraphics[width=0.08\paperwidth]{assets/logo.png}}
% \headerlogoright{\includegraphics[width=0.08\paperwidth]{assets/uvt-logo-fmi.png}}
% \headerlogoright{\qrcode[height=0.06\paperwidth]{https://github.com/alexfikl/uvt-poster}}

% remove logos
\headerlogoleft{}
% \headerlogoright{}


% >>>

% <<< commands

% define some helpful commands
\NewDocumentCommand \dx { O{x} } {\,\mathrm{d} #1}

% >>>

% <<< metadata

\title{Video Over RTP Enhancements and Testing Methods}
\author{Dr. Amiruzzaman \and Huy Nguyen \and Sebastian Tran}
\institute[shortinst]{West Chester University of Pennsylvania}

% add content to the footer

\footerlocation{CSM Fall 2025 Student Poster Session}


% >>>

\begin{document}

% NOTE: each frame will create a new page with a new poster, so you probably do
% not want more of them in a single file.
\begin{frame}[fragile]

% NOTE: this uses a three column format because it is in landscape mode. If you
% are using a portrait mode, two columns is probably better to allow for larger text
\begin{columns}[t]

\separatorcolumn

\begin{column}{0.3\paperwidth}

\begin{block}{Abstract}
   RTP, or real time protocol, is designed for internet transmission with minimal latency. 
   Applications that require near millisecond latency will implement RTP to achieve such goals. 
   Common applications for RTP include real time video streams, online conferencing applications, and voice over IP. 
   To enable additional enhancements and modifications, RTP allows for extensions to the header in order to transmit extra data with the video and audio streams. 
   This study explores the feasibility of various modifications of this extension header and how this impacts data transfer performance.

    \bigskip
    First we attempt to implement sending and receiving Video QOS metrics inside of the RTP packet rather than through the RTP control stream. 
    Second, we add a role based video streaming to ensure video is directed only to authorized users. 
    In addition, we design a testing scheme to measure latency and other QOS metrics in order to compare base protocol designs against other modifications. 
\end{block}

\begin{block}{RTP Packet Header}
    \begin{table}[h!]
        \begin{tabular}{|>{\centering\arraybackslash}p{7mm}|*{32}{>{\centering\arraybackslash}p{3mm}|}}
            \hline
            {\small Bit} & {\tiny 0} & {\tiny 1} & {\tiny 2} & {\tiny 3} & {\tiny 4} & {\tiny 5} & {\tiny 6} & {\tiny 7} & {\tiny 8} & {\tiny 9} & {\tiny 10} & {\tiny 11} & {\tiny 12} & {\tiny 13} & {\tiny 14} & {\tiny 15} & {\tiny 16} & {\tiny 17} & {\tiny 18} & {\tiny 19} & {\tiny 20} & {\tiny 21} & {\tiny 22} & {\tiny 23} & {\tiny 24} & {\tiny 25} & {\tiny 26} & {\tiny 27} & {\tiny 28} & {\tiny 29} & {\tiny 30} & {\tiny 31} \\
            \hline
            {\small 0} & \multicolumn{2}{|c|}{\small V} & {\small P} & {\small X} & \multicolumn{4}{|c|}{\small CC} & {\small M} & \multicolumn{7}{|c|}{\small Payload Type} & \multicolumn{16}{|c|}{\small Sequence Number} \\
            \hline
            {\small 32} & \multicolumn{32}{|c|}{\small Timestamp} \\
            \hline
            {\small 64} & \multicolumn{32}{|c|}{\small SSRC identifier} \\
            \hline
            {\small 96} & \multicolumn{32}{|c|}{\small CSRC identifier(s)} \\
            \hline
            {\small ...} & \multicolumn{16}{|c|}{\small Profile-specific Extension Header Identifier} & \multicolumn{16}{|c|}{\small Extension Header Length}\\
            \hline
            {\small ... } & \multicolumn{32}{|c|}{\small Extension Data} \\
            \hline        
        \end{tabular}
        \caption{RTP Packet Header Layout}
    \end{table}

    \bigskip
    {\small \textbf{V (Version)}: Indicates the verision of the protocol\\}
    \smallskip
    {\small \textbf{P (Padding)}: Tells the reciever that padding is enabled.\\}
    \smallskip
    {\small \textbf{X (Extension)}: Tells the reciver that extension modifications to the header are enabled.\\}
    \smallskip
    {\small \textbf{CC (CSRC Count)}: Stores the number of CSRC identifiers\\}
    \smallskip
    {\small \textbf{M (Marker)}: Used to indicate to the reciver this is special packet\\}
    \smallskip
    {\small \textbf{Payload Type}: Indicates the type of data in the packet\\}
    \smallskip
    {\small \textbf{Sequence Number}: The packet number. Used to correct out of order transmission and detect packet loss\\}
    \smallskip
    {\small \textbf{Timestamp}: Used for synchronization of different streams\\}
    \smallskip
    {\small \textbf{SSRC identifier}: Determines who has sent the packet\\}
    \smallskip
    {\small \textbf{CSRC identifier(s)}: Lists who has contributed data to the packet\\}
    \smallskip
    {\small \textbf{Profile Specific Extension Header Identifier}: Extension specific headers\\}
    \smallskip
    {\small \textbf{Extension Header Length}: Length of the extension data, including header\\}
    \smallskip
    {\small \textbf{Extension Data}: Extra profile specifc data\\}

\end{block}

\begin{block}{Video and Audio Transmission Over RTP}
    Generally, multiplexing video and audio RTP streams over the same session is not recommended.
    If one stream changes protocols, it becomes impossible to determine which one had changed.
    It also provides the ability to particpate in each seperatly. As a result, video and audio streams are sent seperately and require reconstruction. 

    \bigskip

    Streams may be sampled at different rates, and require synchronization. This is done via a common clock, typhically via NTP.
    By combining the data from the NTP from both streams, a common mapping can be found for synchronization

    
\end{block}
\end{column}

\separatorcolumn

\begin{column}{0.3\paperwidth}
\begin{block}{Testing Metrics}

    To test the efficacy of RTP transmissions under different network conditions, we measure these network quality metrics.
    Each has a unique effect on the quality of an RTP stream. To Eliminate overhead of analyzing results during the stream, 
    data will instead be collected and logged to be analyzed later.

    \begin{center}
        \includegraphics[width=0.35\textwidth]{assets/Blank diagram.png}
    \end{center}


    \bigskip

    \begin{alertblock}{Transmission Delay}
        Transmission delay measures the time needed for the sender to push all bits onto the network. 
        \[
            t = \frac{L}{R}
        \]
        \smallskip

        where L is size of packet in bits, and R is bandwidth, measured in bits per second\\

        Transmission delay has a critical role in real time systems, as it propogrates itself as higher latency.
    \end{alertblock}

    \begin{alertblock}{IP Packet Delay Variation (IPDV)}
        IPDV is the can be measured as the root mean square of packet variation samples. 
        \bigskip

        A sample, k, measures the average of the absolute difference in latency between a successive n packets. 
        Where R\textsubscript{i} and S\textsubscript{i} are the arrival times and sending time of the packet respectively

        \[
            IPDV(k) = \frac{\sum_{i = n*(k-1)+2}^{k*n+1}|(R_i - R_{i-1}) - (S_i - S_{i-1})|}{n}
        \]
        \smallskip

        Applying the absolute value, rather than squaring, makes the sample average resistant 
        to a small number of outliers, which are largely unnoticeable to the user.

        \bigskip
        The root mean square of the samples can then be calculated as:

        \[
            IPDV_{RMS}(I) = \sqrt{\frac{\sum_{k = N*(I-1)+1}^{I*N} {IPDV^{2}(K)}}{N}}
        \]
        \smallskip

        Root mean squares provides a measure of variation as a mangnitude, unlike a standard deviation, 
        which measures variation as a distance from the mean.

        \bigskip
        IPDV is noticeable in congested networks. Real time apps compete with other network services for bandwidth. 
        This manifests itself as high packet variation, resulting in a stream that buffers often and 
        offers a lower quality of experience for the user.

    \end{alertblock}

    \begin{alertblock}{Packet Loss}
        Packet loss is the percetage of packets that are successfully received by a client

        \[
            PL = \frac{\text{Packets Sent} - \text{Packets Received}}{\text{Packets Sent}} * 100\ \%
        \]
        \bigskip

        High packet loss causes RTP streams to be choppy and unresponsive. 
        Poor network conditions and failed routing are the main causes of packet loss
        
    \end{alertblock}
\end{block}


\end{column}

\separatorcolumn

\begin{column}{0.3\paperwidth}

\begin{block}{Sessions}
    \begin{center}
        \includegraphics[width=0.5\textwidth]{assets/clients.png}
    \end{center}
\end{block}


\begin{block}{Extensions}

    \begin{center}
        \includegraphics[width=0.5\textwidth]{assets/rtp-rtcp.png}
    \end{center}

    In a traditional RTP setup, data and QoS tranmssions are seperated into different transmissions on seperate ports

    \begin{center}
        \includegraphics[width=0.5\textwidth]{assets/rtp-extension.png}
    \end{center}

    With the extension header, it is possible to transmit both types of data in the header. 
    This adds complexity to RTP in encoding and decoding, but has the benefit of reducing the overhead
    when seperate protocols are used.    

\end{block}

\begin{alertblock}{References}
    \nocite{*}
    \bibliographystyle{plain}
    \bibliography{template}
\end{alertblock}

\end{column}
\separatorcolumn
\end{columns}
\end{frame}

\end{document}